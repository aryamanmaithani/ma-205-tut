\documentclass[12pt]{article}
\usepackage{amsmath, amssymb, amsfonts, amsthm, mathtools}
\usepackage{thmtools}
\usepackage[utf8]{inputenc}
\usepackage[inline]{enumitem}
\usepackage[colorlinks=true]{hyperref}
\setlength\parindent{0pt}

\theoremstyle{definition}
\newtheorem{thm}{Theorem}
\numberwithin{thm}{section}
\newtheorem{lem}[thm]{Lemma}
\newtheorem{defn}[thm]{Definition}
\newtheorem{prop}[thm]{Proposition}
\newtheorem{cor}[thm]{Corollary}
\newtheorem{ex}{Example}


\let\emptyset\varnothing
\newcommand{\id}{\operatorname{id}}
\newcommand{\hint}[1]{\textbf{HIDDEN:} {\color[rgb]{0.95, 0.95, 0.95}#1}}

\pagestyle{plain}

\usepackage{titlesec}
\titleformat{\section}[block]{\sffamily\Large\filcenter\bfseries}{\S\thesection.}{0.25cm}{\Large}
\titleformat{\subsection}[block]{\large\bfseries\sffamily}{\S\S\thesubsection.}{0.2cm}{\large}


\usepackage[a4paper]{geometry}
\usepackage{lipsum}

\usepackage{cleveref}
\crefname{thm}{Theorem}{Theorems}
\crefname{lem}{Lemma}{Lemmas}
\crefname{defn}{Definition}{Definitions}
\crefname{prop}{Proposition}{Propositions}
\crefname{cor}{Corollary}{Corollaries}
\crefname{equation}{}{}

\usepackage{mdframed}
\newenvironment{blockquote}
{\begin{mdframed}[skipabove=0pt, skipbelow=0pt, innertopmargin=4pt, innerbottommargin=4pt, bottomline=false,topline=false,rightline=false, linewidth=2pt]}
{\end{mdframed}}

\usepackage{fancyhdr}
\pagestyle{fancy}
\fancyhf{}
\fancyhead[L]{\sffamily{\S\textbf{\nouppercase{\leftmark}}}}
\fancyhead[R]{\sffamily{\thepage}}

\renewcommand{\familydefault}{\sfdefault}

\title{MA 205: $\mathbb{C}$omplex Analysis\\\large{Extra questions}}
\author{Aryaman Maithani\\\url{https://aryamanmaithani.github.io/tuts/ma-205}}
\date{Autumn Semester 2020-21}

\begin{document}
\maketitle
\setcounter{section}{-1}
\tableofcontents
\newpage\section{Notations} \label{sec:notations}
\begin{enumerate}
	\item $\mathbb{N} = \{1, 2, 3, \ldots\},$ the set of positive integers.
	\item $\mathbb{Z}$ is the set of integers.
	\item $\mathbb{Q}$ is the set of rational numbers.
	\item $\mathbb{R}$ is the set of real numbers.
	% \item Whenever I refer to $\mathbb{R}$ or $\mathbb{R}^n$ or any of its subsets as a metric space, I shall always assume the standard metric unless I explicitly state so otherwise.
	% \item $S^1 = \{\mathbf{x} \in \mathbb{R}^2 \mid \|x\| = 1\} \subset \mathbb{R}^2.$ 
	% \item $\mathbb{S}_{\ge 0} = \{s \in \mathbb{S} \mid s \ge 0\}.$ (So, that defines $\mathbb{Z}_{\ge 0}, \mathbb{Q}_{\ge 0}, \mathbb{R}_{\ge 0}.$)
	% \item $\mathbb{S}^+ = \mathbb{S}_{> 0} = \{s \in \mathbb{S} \mid s > 0\}.$
	\item $A \subset B$ is read as ``$A$ is a subset of $B.$'' In particular, note that $A \subset A$ is true for any set $A.$
	\item $A \subsetneq B$ is read ``$A$ is a \emph{proper} subset of $B.$''
	\item $\supset$ and $\supsetneq$ are defined similarly.
	% \item Given a set $S,$ the set $\mathcal{P}(S)$ is the \emph{power set} of $S,$ i.e., the set of all subsets of $\mathcal{P}(S).$
	\item Given a function $f:X \to Y,$ $A \subset X,$ $B \subset Y,$ we define
	\begin{align*} 
		f(A) &= \{y \in Y \mid y = f(a) \text{ for some } a \in A\} \subset Y,\\
		f^{-1}(B) &= \{x \in X \mid f(x) \in B\} \subset X.
	\end{align*}
	(Note that this $f^{-1}$ is different from the inverse of a function. In particular, this is always defined, even if $f$ is not bijective. However, the $f$ and $f^{-1}$ above need not be ``inverses.'')
	\item A \emph{domain}, as a subset of $\mathbb{C}$ will always refer to a set which is open and path connected.\\
	(Note that this is different from domain of a function.)
	% \item \label{funcrestrict} If $f:X \to Y$ is a function and $A \subset X,$ then $f|_A$ is a function
	% \begin{equation*} 
	% 	f|_A : A \to Y
	% \end{equation*}
	% defined as
	% \begin{equation*} 
	% 	f|_A(a) = f(a), \quad a \in A.
	% \end{equation*}
	% \item Since Rudin follows a non-usual definition for ``countable,'' I shall use the following, which makes it always clear:
	% \begin{enumerate}
	% 	\item At most countable: A set $S$ is at most countable if there exists an injection $i : S \to \mathbb{N}.$
	% 	\item Countably infinite: A set $S$ is countably infinite if it is at most countable and infinite.
	% 	\item Uncountable: A set $S$ is uncountable if it not at most countable.
	% \end{enumerate}
	% In particular, I will not use the term ``countable'' just by itself since Rudin uses it to mean ``countably infinite'' but usually people mean ``at most countable.''
	% %
	% \item Given a set $I,$ $\{P_\alpha\}_{\alpha \in I}$ is a shorthand for writing a set of the form $\{P_\alpha \mid \alpha \in I\}.$ ($P_\alpha$ is defined given the context.)
\end{enumerate}
\newpage\section{Topology}
\begin{enumerate}
	\item Is the interval $(0, 1)$ open as a subset of $\mathbb{C}?$\\
	\hint{No}
	%
	\item Is the interval $(0, 1)$ closed as a subset of $\mathbb{C}?$\\
	\hint{No}
	%
	\item Consider the following four properties that a subset of $\mathbb{C}$ can have:
	\begin{enumerate}
		\item Open
		\item Closed
		\item Bounded
		\item Path connected
	\end{enumerate}
	Thus, we can classify all the subsets of $\mathbb{C}$ into $2^4$ classes on the basis of what properties they have (and what they don't).\\
	Give an example of each or a proof that some certain class cannot have anything.\\
	You may assume that $\emptyset$ and $\mathbb{C}$ are the only subsets of $\mathbb{C}$ which are both open and closed.
	%
	\item Let $U \subset \mathbb{C}$ be open and nonempty. Show that $U$ is not countable.
	%
	\item Let $U \subset \mathbb{C}$ be open and $K$ be countably open. Give examples to show that $U\setminus K$ may or not be open.
\end{enumerate}
\newpage\section{Cauchy Riemann Equations}
\begin{enumerate}
	\item Consider the function $f:\mathbb{C} \to \mathbb{C}$ defined as
	\begin{equation*} 
		f(z) = \bar{z}.
	\end{equation*}
	Show that $f$ is continuous at each point.\\
	Show that $f$ is differentiable at no point.\\
	(This has given us a very easy example of a function which is continuous everywhere but differentiable nowhere. On the contrary, one has to put a lot more effort to construct an example in the case of real analysis.)
	%
	\item Show that the function $f:\mathbb{R}^2\to\mathbb{R}^2$ defined as
	\begin{equation*} 
		f(x, y) = (x, -y)
	\end{equation*}
	is differentiable in the sense that you saw in MA 105. (That is, its total derivative exists at every point.)\\
	Compare this with the previous question.
	%
	\item Let $\Omega$ be open (and not necessarily path-connected). \\
	Let $f:\Omega \to \mathbb{C}$ be holomorphic such that $f'(z) = 0$ for all $z \in \Omega.$\\
	Show that it is \emph{not} necessary that $f$ is constant.\\~\\
	Show that if $\Omega$ is also assumed to be path-connected (that is, $\Omega$ is a domain), then it \emph{is} necessary that $f$ is constant.
	%
	\item Let $\Omega$ be a domain and $f:\Omega \to \mathbb{C}$ be holomorphic.\\
	Suppose
	\begin{equation*} 
		f(z) \in \mathbb{R} \quad \text{for all } z \in \Omega.
	\end{equation*}
	Show that $f$ is constant. (That is, if a complex differentiable function takes only real values, then it must be constant on path-connected sets.)
	%
	\item Let $\Omega$ be a domain and $f:\Omega \to \mathbb{C}$ be holomorphic.\\
	Suppose that $|f|$ is constant. Show that $f$ is constant.
\end{enumerate}
\newpage\section{Series}
\begin{enumerate}
	\item (Cauchy criterion for series.) ``Recall'' Cauchy criterion for convergence from MA 105. (Prove or assume that the analogous thing holds for complex sequences as well.)\\
	Let $(a_n)$ be a sequence of complex numbers. Show that $\displaystyle\sum_{n=1}^{\infty}a_n$ converges iff for every $\epsilon > 0,$ there exists $N \in \mathbb{N}$ such that
	\begin{equation*} 
		\left|\sum_{k=n}^{m}a_n\right| < \epsilon, \quad \text{ for all } m \ge n \ge N.
	\end{equation*}
	%
	\item Let $(a_n)$ be a sequence of complex numbers such that $\sum|a_n|$ converges. Use the above Cauchy criteria to show that $\sum a_n$ converges.
	%
	\item Let $(a_n)$ and $(b_n)$ be complex sequences such that $|a_n| \le |b_n|$ for all $n \in \mathbb{N}.$ Show that if $\sum |b_n|$ converges, then so does $\sum |a_n|$ and hence, so does $\sum a_n.$\\
	Show that you can weaken the ``for all $n \in \mathbb{N}$'' condition to ``for all $n$ sufficiently large.'' (Formulating what we mean by ``sufficiently large'' is part of the exercise.)
	%
	\item Use the above to show that
	\begin{equation*} 
		\sum_{n=1}^{\infty}\dfrac{z^n}{n^2}
	\end{equation*}
	converges for all $z \in \mathbb{C}$ satisfying $|z| = 1.$
	%
	\item Show that
	\begin{equation*} 
		\sum_{n=1}^{\infty}\dfrac{1}{n}
	\end{equation*}
	diverges. \\
	\hint{Compare it with the sequence $1, 1/2, 1/2, 1/4, 1/4, 1/4, 1/4,\ldots.$}
	%
	\item Let $(a_n)$ be a sequence of real numbers and $(b_n)$ a sequence of complex numbers satisfying
	\begin{enumerate}
		\item $(a_n)$ is monotonic,
		\item $\displaystyle\lim_{n\to \infty}a_n = 0,$
		\item there exists $M \ge 0$ such that
		\begin{equation*} 
			\left|\sum_{n=1}^{N}b_n\right| \le M
		\end{equation*}
		for every $N \in \mathbb{N}.$
	\end{enumerate}
	Show that $\displaystyle\sum_{n=1}^{\infty}a_nb_n$ converges.\\~\\
	Here's an outline of what you can do:
	\begin{enumerate}
		\item Define the partial sums $S_n = \displaystyle\sum_{k=1}^{n}a_kb_k$ and $B_n = \displaystyle\sum_{k=1}^{n}b_k.$\\
		Show that
		\begin{equation*} 
			S_n = a_nB_n + \sum_{k=1}^{n-1}B_k(a_k - a_{k+1}).
		\end{equation*}
		(This is called summation by parts.)
		\item Note that $B_n$ is bounded by $M$ and $a_n \to 0.$ Conclude that the first term $\to 0$ as $n \to \infty.$
		\item Note that give any $k,$ we have $|B_k(a_k - a_{k+1})| \le M|a_k - a_{k+1}|.$
		\item Using $(a_n)$ is monotonic, conclude that
		\begin{equation*} 
			\sum_{k=1}^{n-1}|a_k - a_{k+1}| = \sum_{k=1}^{n-1}|a_1 - a_n|.
		\end{equation*}
		\item Conclude that $\displaystyle\lim_{n\to \infty}S_n$ exists.
	\end{enumerate}
	The above is called \textbf{Dirichlet's test}.
	%
	\item Let $z \in \mathbb{C}$ be such that $|z| = 1$ and $z \neq 1.$ Define the sequences $(a_n)$ and $(b_n)$ as
	\begin{equation*} 
		a_n \vcentcolon= \dfrac{1}{n}, \quad b_n \vcentcolon= z^n.
	\end{equation*}
	Show that $(a_n)$ and $(b_n)$ satisfy the hypothesis of Dirichlet's test. Conclude that
	\begin{equation*} 
		\sum_{n=1}^{\infty}\dfrac{z^n}{n}
	\end{equation*}
	converges.
	%
	\item Compute the radius of convergence for the following power series:
	\begin{equation*} 
		\sum_{n=1}^{\infty}\dfrac{z^n}{n}, \quad \sum_{n=1}^{\infty}\dfrac{z^n}{n^2}.
	\end{equation*}
	These should come out to be $1.$ By the previous questions, conclude that the first converges everywhere on the boundary of the disc except at $1.$ However, the second one converges everywhere on the boundary.\\
	Do the same for the power series
	\begin{equation*} 
		\sum_{n=1}^{\infty}z^n.
	\end{equation*}
	\hint{You should get that it converges nowhere on the boundary.}\\
	(Note that these series are (more or less) derivatives and anti-derivatives of each other on the \emph{open} disc. However, they show very different behaviour on the boundary of the disc.)
\end{enumerate}
\newpage\section{Properties of holomorphic functions}
\begin{enumerate}
	\item Let $\mathbb{H} = \{z \in \mathbb{C} : \Re z > 0\}$ be the open right plane.\\
	Construct a non-constant holomorphic function $f:\mathbb{H} \to \mathbb{C}$ such that
	\begin{equation*} 
		f\left(\dfrac{1}{n}\right) = 0, \quad \text{ for all } n \in \mathbb{N}.
	\end{equation*}
	(Does this contradict what we saw in slides? Why not?)
	%
	\item Let $f:\mathbb{C} \to \mathbb{C}$ be a holomorphic function such that
	\begin{equation*} 
		f\left(\dfrac{1}{n}\right) = 0, \quad \text{ for all } n \in \mathbb{N}.
	\end{equation*}
	Show that $f$ is constant (and that the constant is $0$).
\end{enumerate}
\end{document}