\documentclass[12pt]{article}
\usepackage{amsmath, amssymb, amsfonts, amsthm, mathtools}
\usepackage{thmtools}
\usepackage[utf8]{inputenc}
\usepackage[inline]{enumitem}
\usepackage[colorlinks=true]{hyperref}
\setlength\parindent{0pt}

\theoremstyle{definition}
\newtheorem{thm}{Theorem}
\numberwithin{thm}{section}
\newtheorem{lem}[thm]{Lemma}
\newtheorem{defn}[thm]{Definition}
\newtheorem{prop}[thm]{Proposition}
\newtheorem{cor}[thm]{Corollary}
\newtheorem{ex}{Example}


\let\emptyset\varnothing
\newcommand{\id}{\operatorname{id}}
\newcommand{\Res}{\operatorname{Res}}
\newcommand{\hint}[1]{\textbf{HIDDEN:} {\color[rgb]{0.95, 0.95, 0.95}#1}}

\pagestyle{plain}

\usepackage{titlesec}
\titleformat{\section}[block]{\sffamily\Large\filcenter\bfseries}{\S\thesection.}{0.25cm}{\Large}
\titleformat{\subsection}[block]{\large\bfseries\sffamily}{\S\S\thesubsection.}{0.2cm}{\large}


\usepackage[a4paper]{geometry}
\usepackage{lipsum}

\usepackage{cleveref}
\crefname{thm}{Theorem}{Theorems}
\crefname{lem}{Lemma}{Lemmas}
\crefname{defn}{Definition}{Definitions}
\crefname{prop}{Proposition}{Propositions}
\crefname{cor}{Corollary}{Corollaries}
\crefname{equation}{}{}

\usepackage{mdframed}
\newenvironment{blockquote}
{\begin{mdframed}[skipabove=0pt, skipbelow=0pt, innertopmargin=4pt, innerbottommargin=4pt, bottomline=false,topline=false,rightline=false, linewidth=2pt]}
{\end{mdframed}}

\usepackage{fancyhdr}
\pagestyle{fancy}
\fancyhf{}
\fancyhead[L]{\sffamily{\S\textbf{\nouppercase{\leftmark}}}}
\fancyhead[R]{\sffamily{\thepage}}

\renewcommand{\familydefault}{\sfdefault}

\title{MA 205: $\mathbb{C}$omplex Analysis\\\large{Extra questions}}
\author{Aryaman Maithani\\\url{https://aryamanmaithani.github.io/tuts/ma-205}}
\date{Autumn Semester 2020-21}

\begin{document}
\maketitle
\setcounter{section}{-1}
\tableofcontents
\newpage\section{Notations} \label{sec:notations}
\begin{enumerate}
	\item $\mathbb{N} = \{1, 2, 3, \ldots\},$ the set of positive integers.
	\item $\mathbb{Z}$ is the set of integers.
	\item $\mathbb{Q}$ is the set of rational numbers.
	\item $\mathbb{R}$ is the set of real numbers.
	% \item Whenever I refer to $\mathbb{R}$ or $\mathbb{R}^n$ or any of its subsets as a metric space, I shall always assume the standard metric unless I explicitly state so otherwise.
	% \item $S^1 = \{\mathbf{x} \in \mathbb{R}^2 \mid \|x\| = 1\} \subset \mathbb{R}^2.$ 
	% \item $\mathbb{S}_{\ge 0} = \{s \in \mathbb{S} \mid s \ge 0\}.$ (So, that defines $\mathbb{Z}_{\ge 0}, \mathbb{Q}_{\ge 0}, \mathbb{R}_{\ge 0}.$)
	% \item $\mathbb{S}^+ = \mathbb{S}_{> 0} = \{s \in \mathbb{S} \mid s > 0\}.$
	\item $A \subset B$ is read as ``$A$ is a subset of $B.$'' In particular, note that $A \subset A$ is true for any set $A.$
	\item $A \subsetneq B$ is read ``$A$ is a \emph{proper} subset of $B.$''
	\item $\supset$ and $\supsetneq$ are defined similarly.
	% \item Given a set $S,$ the set $\mathcal{P}(S)$ is the \emph{power set} of $S,$ i.e., the set of all subsets of $\mathcal{P}(S).$
	\item Given a function $f:X \to Y,$ $A \subset X,$ $B \subset Y,$ we define
	\begin{align*} 
		f(A) &= \{y \in Y \mid y = f(a) \text{ for some } a \in A\} \subset Y,\\
		f^{-1}(B) &= \{x \in X \mid f(x) \in B\} \subset X.
	\end{align*}
	(Note that this $f^{-1}$ is different from the inverse of a function. In particular, this is always defined, even if $f$ is not bijective. However, the $f$ and $f^{-1}$ above need not be ``inverses.'')
	\item A \emph{domain}, as a subset of $\mathbb{C}$ will always refer to a set which is open and path connected.\\
	(Note that this is different from domain of a function.)
	% \item \label{funcrestrict} If $f:X \to Y$ is a function and $A \subset X,$ then $f|_A$ is a function
	% \begin{equation*} 
	% 	f|_A : A \to Y
	% \end{equation*}
	% defined as
	% \begin{equation*} 
	% 	f|_A(a) = f(a), \quad a \in A.
	% \end{equation*}
	% \item Since Rudin follows a non-usual definition for ``countable,'' I shall use the following, which makes it always clear:
	% \begin{enumerate}
	% 	\item At most countable: A set $S$ is at most countable if there exists an injection $i : S \to \mathbb{N}.$
	% 	\item Countably infinite: A set $S$ is countably infinite if it is at most countable and infinite.
	% 	\item Uncountable: A set $S$ is uncountable if it not at most countable.
	% \end{enumerate}
	% In particular, I will not use the term ``countable'' just by itself since Rudin uses it to mean ``countably infinite'' but usually people mean ``at most countable.''
	% %
	% \item Given a set $I,$ $\{P_\alpha\}_{\alpha \in I}$ is a shorthand for writing a set of the form $\{P_\alpha \mid \alpha \in I\}.$ ($P_\alpha$ is defined given the context.)
\end{enumerate}
\newpage\section{Topology}
\begin{enumerate}
	\item Is the interval $(0, 1)$ open as a subset of $\mathbb{C}?$\\
	\hint{No}
	%
	\item Is the interval $(0, 1)$ closed as a subset of $\mathbb{C}?$\\
	\hint{No}
	%
	\item Consider the following four properties that a subset of $\mathbb{C}$ can have:
	\begin{enumerate}
		\item Open
		\item Closed
		\item Bounded
		\item Path connected
	\end{enumerate}
	Thus, we can classify all the subsets of $\mathbb{C}$ into $2^4$ classes on the basis of what properties they have (and what they don't).\\
	Give an example of each or a proof that some certain class cannot have anything.\\
	You may assume that $\emptyset$ and $\mathbb{C}$ are the only subsets of $\mathbb{C}$ which are both open and closed.
	%
	\item Let $U \subset \mathbb{C}$ be open and nonempty. Show that $U$ is not countable.
	%
	\item Let $U \subset \mathbb{C}$ be open and $K$ be countably open. Give examples to show that $U\setminus K$ may or not be open.
\end{enumerate}
\newpage\section{Cauchy Riemann Equations}
\begin{enumerate}
	\item Consider the function $f:\mathbb{C} \to \mathbb{C}$ defined as
	\begin{equation*} 
		f(z) = \bar{z}.
	\end{equation*}
	Show that $f$ is continuous at each point.\\
	Show that $f$ is differentiable at no point.\\
	(This has given us a very easy example of a function which is continuous everywhere but differentiable nowhere. On the contrary, one has to put a lot more effort to construct an example in the case of real analysis.)
	%
	\item Show that the function $f:\mathbb{R}^2\to\mathbb{R}^2$ defined as
	\begin{equation*} 
		f(x, y) = (x, -y)
	\end{equation*}
	is differentiable in the sense that you saw in MA 105. (That is, its total derivative exists at every point.)\\
	Compare this with the previous question.
	%
	\item Let $\Omega$ be open (and not necessarily path-connected). \\
	Let $f:\Omega \to \mathbb{C}$ be holomorphic such that $f'(z) = 0$ for all $z \in \Omega.$\\
	Show that it is \emph{not} necessary that $f$ is constant.\\~\\
	Show that if $\Omega$ is also assumed to be path-connected (that is, $\Omega$ is a domain), then it \emph{is} necessary that $f$ is constant.
	%
	\item Let $\Omega$ be a domain and $f:\Omega \to \mathbb{C}$ be holomorphic.\\
	Suppose
	\begin{equation*} 
		f(z) \in \mathbb{R} \quad \text{for all } z \in \Omega.
	\end{equation*}
	Show that $f$ is constant. (That is, if a complex differentiable function takes only real values, then it must be constant on path-connected sets.)
	%
	\item Let $\Omega$ be a domain and $f:\Omega \to \mathbb{C}$ be holomorphic.\\
	Suppose that $|f|$ is constant. Show that $f$ is constant.
\end{enumerate}
\newpage\section{Series}
\begin{enumerate}
	\item (Cauchy criterion for series.) ``Recall'' Cauchy criterion for convergence from MA 105. (Prove or assume that the analogous thing holds for complex sequences as well.)\\
	Let $(a_n)$ be a sequence of complex numbers. Show that $\displaystyle\sum_{n=1}^{\infty}a_n$ converges iff for every $\epsilon > 0,$ there exists $N \in \mathbb{N}$ such that
	\begin{equation*} 
		\left|\sum_{k=n}^{m}a_n\right| < \epsilon, \quad \text{ for all } m \ge n \ge N.
	\end{equation*}
	%
	\item Let $(a_n)$ be a sequence of complex numbers such that $\sum|a_n|$ converges. Use the above Cauchy criteria to show that $\sum a_n$ converges.
	%
	\item Let $(a_n)$ and $(b_n)$ be complex sequences such that $|a_n| \le |b_n|$ for all $n \in \mathbb{N}.$ Show that if $\sum |b_n|$ converges, then so does $\sum |a_n|$ and hence, so does $\sum a_n.$\\
	Show that you can weaken the ``for all $n \in \mathbb{N}$'' condition to ``for all $n$ sufficiently large.'' (Formulating what we mean by ``sufficiently large'' is part of the exercise.)
	%
	\item Use the above to show that
	\begin{equation*} 
		\sum_{n=1}^{\infty}\dfrac{z^n}{n^2}
	\end{equation*}
	converges for all $z \in \mathbb{C}$ satisfying $|z| = 1.$
	%
	\item Show that
	\begin{equation*} 
		\sum_{n=1}^{\infty}\dfrac{1}{n}
	\end{equation*}
	diverges. \\
	\hint{Compare it with the sequence $1, 1/2, 1/2, 1/4, 1/4, 1/4, 1/4,\ldots.$}
	%
	\item Let $(a_n)$ be a sequence of real numbers and $(b_n)$ a sequence of complex numbers satisfying
	\begin{enumerate}
		\item $(a_n)$ is monotonic,
		\item $\displaystyle\lim_{n\to \infty}a_n = 0,$
		\item there exists $M \ge 0$ such that
		\begin{equation*} 
			\left|\sum_{n=1}^{N}b_n\right| \le M
		\end{equation*}
		for every $N \in \mathbb{N}.$
	\end{enumerate}
	Show that $\displaystyle\sum_{n=1}^{\infty}a_nb_n$ converges.\\~\\
	Here's an outline of what you can do:
	\begin{enumerate}
		\item Define the partial sums $S_n = \displaystyle\sum_{k=1}^{n}a_kb_k$ and $B_n = \displaystyle\sum_{k=1}^{n}b_k.$\\
		Show that
		\begin{equation*} 
			S_n = a_nB_n + \sum_{k=1}^{n-1}B_k(a_k - a_{k+1}).
		\end{equation*}
		(This is called summation by parts.)
		\item Note that $B_n$ is bounded by $M$ and $a_n \to 0.$ Conclude that the first term $\to 0$ as $n \to \infty.$
		\item Note that give any $k,$ we have $|B_k(a_k - a_{k+1})| \le M|a_k - a_{k+1}|.$
		\item Using $(a_n)$ is monotonic, conclude that
		\begin{equation*} 
			\sum_{k=1}^{n-1}|a_k - a_{k+1}| = \sum_{k=1}^{n-1}|a_1 - a_n|.
		\end{equation*}
		\item Conclude that $\displaystyle\lim_{n\to \infty}S_n$ exists.
	\end{enumerate}
	The above is called \textbf{Dirichlet's test}.
	%
	\item Let $z \in \mathbb{C}$ be such that $|z| = 1$ and $z \neq 1.$ Define the sequences $(a_n)$ and $(b_n)$ as
	\begin{equation*} 
		a_n \vcentcolon= \dfrac{1}{n}, \quad b_n \vcentcolon= z^n.
	\end{equation*}
	Show that $(a_n)$ and $(b_n)$ satisfy the hypothesis of Dirichlet's test. Conclude that
	\begin{equation*} 
		\sum_{n=1}^{\infty}\dfrac{z^n}{n}
	\end{equation*}
	converges.
	%
	\item Compute the radius of convergence for the following power series:
	\begin{equation*} 
		\sum_{n=1}^{\infty}\dfrac{z^n}{n}, \quad \sum_{n=1}^{\infty}\dfrac{z^n}{n^2}.
	\end{equation*}
	These should come out to be $1.$ By the previous questions, conclude that the first converges everywhere on the boundary of the disc except at $1.$ However, the second one converges everywhere on the boundary.\\
	Do the same for the power series
	\begin{equation*} 
		\sum_{n=1}^{\infty}z^n.
	\end{equation*}
	\hint{You should get that it converges nowhere on the boundary.}\\
	(Note that these series are (more or less) derivatives and anti-derivatives of each other on the \emph{open} disc. However, they show very different behaviour on the boundary of the disc.)
	%
	\item Let $(a_n)$ and $(b_n)$ be sequences of complex numbers such that the power series
	\begin{equation*} 
		\sum_{n=0}^{\infty}a_nz^n \quad \text{and} \quad \sum_{n=0}^{\infty}b_nz^n
	\end{equation*}
	have radii of convergence $R_1$ and $R_2$ respectively.\\
	Show that if $R_1 < R_2,$ then the radius of convergence of
	\begin{equation*} 
		\sum_{n=0}^{\infty}(a_n + b_n)z^n
	\end{equation*}
	is $R_1.$\\
	Show that if $R_1 = R_2,$ then all that we can conclude is that the radius of convergence of the sum is at least $R_1.$\\
	(The possibilities of radii being $0$ or $\infty$ should not be excluded.)\\
	At this point, I'll remark that you should recall that the radius of convergence being $R$ not only says that it converges for all $|z| < R$ but also that it \emph{diverges} for all $|z| > R.$
\end{enumerate}
\newpage\section{Properties of holomorphic functions}
\begin{enumerate}
	\item Let $\mathbb{H} = \{z \in \mathbb{C} : \Re z > 0\}$ be the open half right plane.\\
	Construct a non-constant holomorphic function $f:\mathbb{H} \to \mathbb{C}$ such that
	\begin{equation*} 
		f\left(\dfrac{1}{n}\right) = 0, \quad \text{ for all } n \in \mathbb{N}.
	\end{equation*}
	(Does this contradict what we saw in slides? Why not?)
	%
	\item Let $f:\mathbb{C} \to \mathbb{C}$ be a holomorphic function such that
	\begin{equation*} 
		f\left(\dfrac{1}{n}\right) = 0, \quad \text{ for all } n \in \mathbb{N}.
	\end{equation*}
	Show that $f$ is constant (and that the constant is $0$).\\
	Compare this with the previous question.
	%
	\item Suppose that the domain in the previous question was replaced by an arbitrary domain $\Omega$ such that $\{n^{-1} : n \in \mathbb{N}\} \subset \Omega.$\\
	Characterise $\Omega$ precisely such that the above $f(1/n) = 0$ condition ensures that $f$ is constant. (That is, come up with a rule such that if $\Omega$ follows that rule, then $f$ has to be constant and that if $\Omega$ does not follow the rule, then $f$ may be non-constant.)\\
	\hint{The rule should be (equivalent to): $0 \in \Omega.$}
	%
	\item Let $f, g:\mathbb{C} \to \mathbb{C}$ be holomorphic functions which are nonzero everywhere. Suppose that $f$ and $g$ satisfy
	\begin{equation*} 
		\left(\dfrac{f'}{f}\right)\left(\dfrac{1}{n}\right) = \left(\dfrac{g'}{g}\right)\left(\dfrac{1}{n}\right), \quad \text{ for all } n \in \mathbb{N}.
	\end{equation*}
	(The LHS is the function $f'/f$ is evaluated at $1/n$ and similarly for the RHS.)\\
	Find a simpler relation between $f$ and $g.$ (Yes, ``simpler'' is subjective.)
	\item Consider the principal branch $\log:\mathbb{C}\setminus(-\infty, 0] \to \mathbb{C}.$ Choose the point $z_0 = -3 + 4\iota$ in the domain and expand $\log$ as a power series around this point.\\
	Show that the radius of convergence of this power series is $5$ and not $4.$
\end{enumerate}
%
\newpage\section{Picard, Rouch\'{e}, Cauchy's estimates, Liouville, MMT}
\begin{enumerate}
	\item Show that $\exp(z) = z$ has a solution in $\mathbb{C}.$
	%
	\item Let $f, g$ be entire functions such that $\exp f + \exp g = 1.$ Show that $f$ and $g$ are constant.
	%
	\item Let $f$ be a non-vanishing entire function. (That is, $f$ is never zero.) Show that there exists an entire function $g$ such that $f = \exp\circ g.$
	%
	\item Let $f$ be a non-vanishing entire function. (That is, $f$ is never zero.) Show that there exists an entire function $g$ such that $f = g^2.$ (That is, $f(z) = (g(z))^2$ for all $z \in \mathbb{C}.$)
	%
	\item \textbf{Minimum Modulus Theorem.} \\
	Let $\Omega$ be open and connected and $f:\Omega\to\mathbb{C}$ be non-constant and non-vanishing. Show that $|f|$ attains no minimum.
	%
	\item Without using Little Picard, show that there is no entire non-constant function such that the image is contained in the upper half plane.\\
	\hint{Consider $z \mapsto \dfrac{z - \iota}{z + \iota}.$}
	\item Let $P(z)$ and $Q(z)$ be polynomials with real coefficients such that $\deg Q(z) \ge \deg P(z) + 2.$ \\
	Moreover, assume that $Q$ has no real root.
	\begin{enumerate}
		\item Show that there exist constants $C, R > 0$ such that
		\begin{equation*} 
			\left|\dfrac{P(z)}{Q(z)}\right| \le \dfrac{C}{|z|^2}
		\end{equation*}
		for all $z \in \mathbb{C}$ with $|z| > R.$
		\item Conclude the improper integrals
		\begin{equation*} 
			\int_{-\infty}^{-R} \dfrac{P(x)}{Q(x)} {\mathrm{d}}x \quad\text{and}\quad \int_{R}^{\infty} \dfrac{P(x)}{Q(x)} {\mathrm{d}}x
		\end{equation*}
		exist.
		\item Argue that the integral
		\begin{equation*} 
			\int_{-R}^{R} \dfrac{P(x)}{Q(x)} {\mathrm{d}}x
		\end{equation*}
		also exists.
		\item Conclude that the integral
		\begin{equation*} 
			\int_{-\infty}^{\infty} \dfrac{P(x)}{Q(x)} {\mathrm{d}}x
		\end{equation*}
		exists.
		\item Let $\gamma_r$ denote the semicircle (without the diameter) in the upper half plane with ends $-r$ and $r.$ Show that 
		\begin{equation*} 
			\lim_{r\to \infty}\int_{\gamma_r}^{} \dfrac{P(z)}{Q(z)} {\mathrm{d}}z = 0.
		\end{equation*}
		\item Use Cauchy residue theorem to conclude that $\displaystyle\dfrac{1}{2\pi\iota}\int_{-\infty}^{\infty} \dfrac{P(x)}{Q(x)} {\mathrm{d}}x$ is equal to the sum of the residues of $p(x)/q(x)$ at the poles in the upper half plane.
	\end{enumerate}
	\item Let $f:\mathbb{C}^\times\to\mathbb{C}$ be a holomorphic function such that 
	\begin{equation*} 
		|f(z)| \le \sqrt{|z|} + \dfrac{1}{\sqrt{|z|}}
	\end{equation*}
	for all $z \in \mathbb{C}^\times.$ 
	\begin{enumerate}
		\item Show that $0$ is a removable singularity of $f.$ Conclude that $f$ can be made entire.
		\item Show that $\infty$ is a removable singularity of $f.$ Conclude that $f$ is bounded.
		\item Conclude that $f$ is constant.
	\end{enumerate}
	%
	\item Let $U = \{z \in \mathbb{C} : |z| < 1\}.$ For $\alpha \in U,$ define
	\begin{equation*} 
		\varphi_\alpha(U) = \dfrac{z - \alpha}{1 - \bar{\alpha}z}.
	\end{equation*}
	This function is defined and holomorphic on $\mathbb{C}\setminus\{\bar{\alpha}^{-1}\}.$ In particular, it is holomorphic on $U.$
	\begin{enumerate}
		\item If $\alpha \in U,$ show that $-\alpha \in U.$ Show that $\varphi_{-\alpha}(\varphi_\alpha(z)) = z$ for all $z$ in the domain. Conclude that $\varphi_\alpha$ is one-one.
		\item Show that if $|z| = 1,$ then $\left|\varphi_\alpha(z)\right| = 1.$
		\item Show that $\varphi_\alpha$ is nonconstant. Conclude that if $z \in U,$ then $\varphi_\alpha(z) \in U.$\\
		\hint{Use MMT.}
		\item The above shows that $\varphi_\alpha(U) \subset U.$ By considering $\varphi_{-\alpha},$ show that the equality $\varphi_\alpha(U) = U$ is true. Conclude that $\varphi_\alpha|_U$ is a bijection from $U$ onto itself.
	\end{enumerate}
	%
	\item Suppose $f, g$ are entire functions and $|f(z)| \le |g(z)|$ for every $z \in \mathbb{C}.$ What conclusion can you draw about $f$ and $g$? \\
	\hint{If $g$ is not identically zero, then its zeroes are isolated. Show that all zeroes of $g$ are actually removable singularities of $f/g.$ Thus, conclude that $f/g$ is entire. Finish it from that. }
	%
	\item Suppose $f$ is an entire function and there exist constants $A, B > 0$ and $k \in \mathbb{N}$ such that 
	\begin{equation*} 
		|f(z)| \le A + B|z|^k
	\end{equation*}
	for all $z \in \mathbb{C}.$ Show that $f$ is a polynomial of degree at most $k.$
	%
	\item \textbf{Fractional Residue Theorem.} \\
	Let $f$ have a simple pole at $z_0.$ Let $\delta > 0$ be such that $f$ is holomorphic on the punctured neighbourhood $B_\delta(z_0)\setminus\{z_0\}.$\\
	Fix $\alpha \in (0, 2\pi]$ and $\alpha_0 \in [0, 2\pi).$ \\
	For $0 < r < \delta,$ define $\gamma_r(\theta) \vcentcolon= z_0 + re^{\iota(\theta + \alpha_0)}$ for $\theta \in [0, \alpha].$ (Draw a picture to see that this is an arc centered at $z_0$ subtending angle $\alpha$ and having radius $r.$)

	Let $l \vcentcolon= \Res(f; z_0).$ \begin{enumerate}
		\item Show that $g(z) \vcentcolon= f(z) - \dfrac{l}{z - z_0}$ is holomorphic on $B_\delta(z_0).$\\
		(More correctly: show that $z_0$ is a removable singularity of $g.$)
		\item Conclude that there exists $M$ such that $|g(z)| \le M$ for $z \in B_\delta(z_0).$
		\item Conclude that
		\begin{equation*} 
			\lim_{r\to 0}\int_{\gamma_r}^{} g(z) {\mathrm{d}}z = 0.
		\end{equation*}
		\item Conclude that
		\begin{equation*} 
			\lim_{r\to 0}\int_{\gamma_r}^{} f(z) {\mathrm{d}}z = \lim_{r\to 0}\int_{\gamma_r}^{} \dfrac{l}{z - z_0} {\mathrm{d}}z.
		\end{equation*}
		\item Show that the RHS is $\alpha\iota\Res(f; z_0)$ and conclude the fractional residue theorem.
	\end{enumerate} 

	\item Let $f:\Omega\to\mathbb{C}$ be holomorphic. Recall that a fixed point of $f$ is a point $z_0 \in \Omega$ such that $f(z_0) = z_0.$ Suppose that $\Omega$ contains the closed unit disc. Moreover, assume that $|f(z)| < 1$ for $|z| = 1.$ Show that $f$ has no fixed points in the open unit disc.
	\item Suppose $f:\Omega\to\mathbb{C}$ is holomorphic and $\Omega$ contains the closed unit disc. Suppose that $f(0) = 1$ and $|f(z)| > 2$ if $|z| = 1.$ Then, show that $f$ has at least one zero in the open unit disc.\\
	\hint{Minimum modulus theorem.}
\end{enumerate}
\end{document}