\documentclass[12pt]{article}
\usepackage{amsmath, amssymb, amsfonts, amsthm, mathtools,mathrsfs}
\usepackage{thmtools}
\usepackage[utf8]{inputenc}
\usepackage[inline]{enumitem}
\usepackage[colorlinks=true]{hyperref}
\usepackage{tikz-cd}
\usepackage{witharrows}
\usepackage{datetime2}

\setlength\parindent{0pt}

\theoremstyle{definition}
\newtheorem{thm}{Theorem}
\numberwithin{thm}{section}
\newtheorem{lem}[thm]{Lemma}
\newtheorem{defn}[thm]{Definition}
\newtheorem{prop}[thm]{Proposition}
\newtheorem{cor}[thm]{Corollary}
\newtheorem{ex}{Example}


\let\emptyset\varnothing

\pagestyle{plain}

\usepackage{titlesec}
\titleformat{\section}[block]{\sffamily\Large\filcenter\bfseries}{\S\thesection.}{0.25cm}{\Large}
\titleformat{\subsection}[block]{\large\bfseries\sffamily}{\S\S\thesubsection.}{0.2cm}{\large}

\usepackage[a4paper]{geometry}
\usepackage{lipsum}

\usepackage{cleveref}
\crefname{thm}{Theorem}{Theorems}
\crefname{lem}{Lemma}{Lemmas}
\crefname{defn}{Definition}{Definitions}
\crefname{prop}{Proposition}{Propositions}
\crefname{cor}{Corollary}{Corollaries}
\crefname{equation}{}{}

\usepackage{mdframed}
\newenvironment{blockquote}
{\begin{mdframed}[skipabove=0pt, skipbelow=0pt, innertopmargin=4pt, innerbottommargin=4pt, bottomline=false,topline=false,rightline=false, linewidth=2pt]}
{\end{mdframed}}
\newenvironment{soln}{\begin{proof}[Solution]}{\end{proof}}

\usepackage{fancyhdr}
\pagestyle{fancy}
\fancyhf{}
\fancyhead[L]{\sffamily{\S\textbf{\nouppercase{\leftmark}}}}
\fancyhead[R]{\sffamily{\thepage}}

% \usepackage{xcolor}
% \definecolor{mybgcolor}{RGB}{50, 50, 50} %46, 51, 63
% \usepackage{pagecolor}
% \pagecolor{mybgcolor}
% \color{white}
% \mdfsetup{backgroundcolor=mybgcolor, fontcolor=white}

\renewcommand{\familydefault}{\sfdefault}

\title{MA 205: Complex Analysis\\\large{Tutorial Solutions}}
\author{Aryaman Maithani\\\url{https://aryamanmaithani.github.io/tuts/ma-205}}
\date{Autumn Semester 2020-21\\~\\Last update: \DTMnow}

\begin{document}
\maketitle
\setcounter{section}{-1}
\section{Notations}
\begin{enumerate}
	\item Given $z \in \mathbb{C},$ $\Re z$ and $\Im z$ will denote the real and imaginary parts of $z,$ respectively.
	\item Given $z \in \mathbb{C},$ $\bar{z}$ will denote the complex conjugate of $z.$
	\item Given $z \in \mathbb{C},$ $\left|z\right|$ will denote the modulus of $z,$ defined as $\sqrt{z\bar{z}}$ or $\sqrt{\left(\Re z\right)^2 + \left(\Im z\right)^2}.$
\end{enumerate}
\newpage\section{Tutorial 1}
\begin{center}
	25th August, 2020
\end{center}
\textbf{Notation:} The set $\mathbb{C}[x]$ is the set of all polynomials (with indeterminate $x$) with complex coefficients. Similarly, $\mathbb{R}[x]$ is defined.
\begin{enumerate}
	\item  Show that complex polynomial of degree $n$ has exactly $n$ roots. (Assuming fundamental theorem of algebra.)\\
	Remark (my own): The above is counting the roots \emph{with} multiplicity. That is, if $f(z) = (z - \iota)^2(z - 2),$ then $\iota$ is counted twice and $2$ once.
	\begin{soln}
		Let $f(x) \in \mathbb{C}[x]$ be a polynomial of degree $n.$
		We prove this via induction on $n.$\\
		$n = 1.$ Then, $f(x) = a_0 + a_1x$ for some $a_0, a_1 \in \mathbb{C}$ and $a_1 \neq 0.$\\
		Note that
		\begin{align*} 
			f(x) &= 0\\
			\iff a_0 + a_1x &= 0\\
			\iff a_1x &= -a_0\\
			\iff x &= -\dfrac{a_0}{a_1}.
		\end{align*}
		Thus, $f(x)$ has exactly $1$ root.\\~\\
		Let us assume that whenever $g(x) \in \mathbb{C}[x]$ is a polynomial of degree $n,$ then $g(x)$ has exactly $n$ roots. (Counted with multiplicity.)\\
		Let $f(x) \in \mathbb{C}[x]$ be a polynomial of degree $n + 1.$ By FTA, there exists a root $x_0 \in \mathbb{C}.$ Thus, we can write
		\begin{equation*} 
			f(x) = (x - x_0)g(x)
		\end{equation*}
		for some polynomial $g(x) \in \mathbb{C}[x]$ of degree $n.$ Moreover, note that 
		\begin{equation*} 
			f(x) = 0 \iff x = x_0 \text{ or } g(x) = 0.
		\end{equation*}
		By induction, the latter is possible for exactly $n$ values of $x.$ Thus, in total, $f(x)$ has $n + 1$ roots. (Both counts are with multiplicity.)
	\end{soln}
	%
	%
	%
	\item Show that a real polynomial that is irreducible has degree at most two. i.e., if
	\begin{equation*} 
		f(x) = a_0 + a_1x + \cdots + a_nx^n, \quad a_i \in \mathbb{R}
	\end{equation*}
	then there are non-constant real polynomials $g$ and $h$ such that $f(x) = g(x)h(x)$ if $n \ge 3.$\\
	Remark (my own): $a_n \neq 0,$ of course.
	\begin{soln}
		Let $f(x) \in \mathbb{R}[x]$ with degree $\ge 3$ as above.\\
		If $f(x)$ has a real root, then we are done by factoring as in the earlier question.\\~\\
		Thus, let us assume that $f(x) = 0$ has no real solution.\\
		We may view $f(x) \in \mathbb{C}[x].$ Now, using FTA, we know that $f(x)$ has a complex root $x_0 \in \mathbb{C}.$ By assumption, we must have $x_0 \notin \mathbb{R}$ or that $x_0 \neq \overline{x_0}.$ \\
		\begin{blockquote}
			\textbf{Claim.} $f(\overline{x_0}) = 0.$
			\begin{proof} 
				Note that

				\[\begin{WithArrows}[displaystyle]
			    f(\overline{x_0}) &= a_0 + a_1\overline{x_0} + \cdots + a_n(\overline{x_0})^n \Arrow{$\because \overline{z^n} = \bar{z}^n$}\\
					&= a_0 + a_1\overline{x_0} + \cdots + a_n\overline{x_0^n} \Arrow{$\because a_i \in \mathbb{R}$ and so, $a_i = \overline{a_i}$}\\
					&= \overline{a_0} + \overline{a_1}\;\overline{x_0} + \cdots + \overline{a_n}\overline{x_0^n} \Arrow{$\overline{z_1z_2 + z_3} = \overline{z_1}\;\overline{z_2} + \overline{z_3}$}\\
					&= \overline{f(x_0)}\\
					&= \bar{0}\\
					&= 0
			  \end{WithArrows}\]
			\end{proof}
		\end{blockquote}
		Define $g(x) = (x - x_0)(x - \overline{x_0}).$ A priori, this is a polynomial in $\mathbb{C}[x].$ However, upon multiplication, we see that the polynomial is actually an element of $\mathbb{R}[x].$ Indeed, we have
		\begin{equation*} 
			(x - x_0)(x - \overline{x_0}) = (x^2 - (2\Re x_0)x + |x_0|^2) \in \mathbb{R}[x].	
		\end{equation*}
		By our claim, we see that $g(x)$ divides $f(x)$ in $\mathbb{C}[x].$ (Since $x_0$ and $\overline{x_0}$ are distinct, the polynomials $x - x_0$ and $x - \overline{x_0}$ are ``coprime'' and thus, if they individually divide $f(x),$ then their product must too.) \\
		Thus,
		\begin{equation*} 
			f(x) = g(x)h(x)
		\end{equation*}
		for some $h(x) \in \mathbb{C}[x].$ However, since $f(x)$ and $g(x)$ are both real polynomials, so is $h(x).$ \hfill (Why?)\\
		Thus, we get that
		\begin{equation*} 
			f(x) = g(x)h(x)
		\end{equation*}
		for real polynomials $g(x)$ and $h(x).$ Moreover, note that $\deg g(x) = 2$ and $\deg h(x) = n - 2 \ge 1.$ Thus, both are non-constant.
	\end{soln}
	%\\
	%\\
	%
	\item Show that if $U$ is a path connected open set in $\mathbb{C},$ so is $U$ minus any finite set.
	%
	\begin{soln}
		We will first prove the following claim:
		\begin{blockquote}
			\textbf{Claim:} Let $U \subset \mathbb{C}$ be open and $w \in U.$ Then, $U \setminus \{w\}$ is open.
			\begin{proof} 
				Let $z_0 \in U\setminus\{w\}$ be arbitrary. Since $U$ was open, there exists $\delta_1 > 0$ such that
				\begin{equation*} 
					B_{\delta_1}(z_0) \subset U.
				\end{equation*}
				Since $z_0 \neq w,$ we have that $\delta_2 \vcentcolon= |z_0 - w| > 0.$\\
				Choose $\delta := \min\{\delta_1, \delta_2\}.$ Clearly, $\delta > 0.$ Moreover, we have
				\begin{equation*} 
					w \notin B_{\delta_2}(z_0) \supset B_{\delta}(z_0)
				\end{equation*}
				and thus, $w \notin B_{\delta}(z_0).$ Also,
				\begin{equation*} 
					B_{\delta}(z_0) \subset B_{\delta_1}(z_0) \subset U.
				\end{equation*}
				Thus, we get that
				\begin{equation*} 
					B_{\delta}(z_0) \subset U \setminus \{w\},
				\end{equation*}
				proving that $U\setminus\{w\}$ is open.
			\end{proof}
		\end{blockquote}
		By the above proof, we see that removing one point from an open set keeps it open. Thus, if we show that removing one point from an open path-connected set leaves it path-connected, then we are done since we can induct to get any other \textbf{finite}\footnote{Finiteness is important. Induction cannot prove this result for a countably infinite set.} set.\\~\\
		Thus, we now show that if $U$ is open and path-connected, so is $U\setminus\{w\}.$ (Where $w \in U$ is any arbitrary element.)\\~\\
		Let $z_0, z_1 \in U\setminus\{w\}.$ We wish to show that there is a path in $U\setminus\{w\}$ connecting $z_0$ to $z_1.$\\
		Since $U$ was path-connected to begin with, there exists a path $\sigma:[0, 1] \to U$ such that
		\begin{equation*} 
			\sigma(0) = z_0, \quad \sigma(1) = z_1.
		\end{equation*}
		If $\sigma(x) \neq w$ for any $x \in [0, 1],$ then we are done since $\sigma$ is a path in $U\setminus\{w\}$ as well.\\
		Suppose that this is not the case.\\
		Then, we choose a $\delta > 0$ such that the \emph{closed} ball
		\begin{equation*} 
			B \vcentcolon= \{z \in \mathbb{C} : |z - w| \le \delta\}
		\end{equation*}
		has the following properties:
		\begin{enumerate}
			\item $z_0 \notin B,$
			\item $z_1 \notin B,$
			\item $B \subset U.$
		\end{enumerate}
		(Why must such a $\delta$ exist? There exists a $\delta_1$ for which we get the first two properties since $z_0$ and $z_1$ are distinct from $w.$ For the last property, let $\delta_2$ be any such that $B_{\delta_2}(w) \subset U,$ which exists since $U$ is open. Then, consider $\delta_2/2.$ The \emph{closed} ball of this radius must again be completely within $U.$ Take the minimum of $\delta_1$ and $\delta_2/2$.)\\~\\
		Note that
		\begin{equation*} 
			\sigma^{-1}(B) = \{x \in [0, 1] : \sigma(x) \in B\}
		\end{equation*}
		is nonempty since $w \in \sigma^{-1}(B).$ Moreover, it must be closed. (Why?)\\
		Since it is a subset of $[0, 1],$ it is clearly bounded. Define
		\begin{equation*} 
			s \vcentcolon= \inf \sigma^{-1}(B), \quad t \vcentcolon= \sup\sigma^{-1}(B).
		\end{equation*}
		Since the set is closed, both $s$ and $t$ are elements of $\sigma^{-1}(B).$ Note that $\sigma(0) \notin B$ and $\sigma(1) \notin B$ and thus,
		\begin{equation*} 
			0 < s < t < 1.
		\end{equation*}
		(Why is the inequality $s < t$ strict?)\\
		Note that $\sigma(s)$ and $\sigma(t)$ must lie on the circumference of $B.$ (Why?) (This also shows why $s < t.$)\\
		Now consider the path $\sigma':[0, 1] \to U$ defined as follows:
		\begin{equation*} 
			\sigma'(x) = \begin{cases}
				\sigma(x) & \text{if } x \in [0, s] \cup [t, 1]\\
				\gamma(x) & \text{if } x \in [s, t],
			\end{cases}
		\end{equation*}	
		where $\gamma:[s, t] \to B$ is the path which is the arc joining $\sigma(s)$ to $\sigma(t).$ (Note that $\sigma(s) = \sigma(t)$ is possible in which case, it's the constant path.)\\
		Clearly, $\sigma'$ avoids $w$ and is continuous. \hfill (Why?)\\~\\
		Moreover, $\sigma'(0) = \sigma(0) = z_0$ and $\sigma'(1) = \sigma(1) = z_1$ and thus, $\sigma'$ is a path from $z_0$ to $z_1$ in $U \setminus \{w\},$ showing that $U\setminus\{w\}$ is path-connected.
	\end{soln}
	%\\
	%\\
	%\\
	%
	\item Check for real differentiability and holomorphicity:
	\begin{enumerate}
		\item $f(z) = c,$
		\item $f(z) = z,$
		\item $f(z) = z^n,$ $n \in \mathbb{Z},$
		\item $f(z) = \Re z,$
		\item $f(z) = \left|z\right|,$
		\item $f(z) = \left|z\right|^2,$
		\item $f(z) = \bar{z},$
		\item $f(z) = \begin{cases}
		\dfrac{z}{\bar{z}} & \text{if } z \neq 0,\\
		0 & \text{if } z = 0.
		\end{cases}$
	\end{enumerate}
	%
	\begin{soln}
		Not going to do all.
		\begin{enumerate}
			\item Real differentiable and holomorphic, both.
			\item Real differentiable and holomorphic, both.
			\item Real differentiable and holomorphic, both. Let us see why.\\
			As we know, holomorphicity implies real differentiability, so we only check that $f$ is holomorphic on $\mathbb{C}.$\\
			Let $z_0 \in \mathbb{C}$ be arbitrary. We show that the limit
			\begin{equation*} 
				\lim_{z\to z_0}\dfrac{f(z) - f(z_0)}{z - z_0}
			\end{equation*}
			exists.\\
			This is clear because for $z_0 \neq z,$ we have
			\begin{equation*} 
				\dfrac{z^n - z_0^n}{z - z_0} = \sum_{k=0}^{n-1}z^kz_0^{n - 1 - k}.
			\end{equation*}
			The limit $z \longrightarrow z_0$ of the RHS clearly exists.
			\item Real differentiable but not holomorphic. Note that $f$ can be written as
			\begin{equation*} 
				f(x + \iota y) = x + 0\iota.
			\end{equation*}
			Thus, $u(x, y) = x$ and $v(x, y) = 0.$\\
			This is clearly real differentiable everywhere since all the partial derivatives exist everywhere and are continuous.\\
			However, we show that $f$ is not complex differentiable at any point. Thus, it is not holomorphic.\\
			This is easy because one sees that $u_x(x_0, y_0) = 1$ and $v_y(x_0, y_0) = 0$ for all $(x_0, y_0) \in \mathbb{R}^2$ and thus, the CR equations don't hold.
			\item $|z|$ is real differentiable everywhere except $0$ and complex differentiable nowhere. Breaking the function as earlier, we have
			\begin{equation*} 
				u(x, y) = \sqrt{x^2 + y^2}, \quad v(x, y) = 0.
			\end{equation*}
			On $\mathbb{R}^2\setminus\{(0, 0)\},$ all partial derivatives exist and are continuous. At $(0, 0),$ $u_x$ and $u_y$ fail to exist.\\~\\
			This clearly shows that $f$ is not complex differentiable at $0 \in \mathbb{C}$ since it is not even real differentiable there.\\
			However, we see that $v_y = 0 = v_x$ everywhere else but at least one of $u_x$ or $u_y$ is nonzero on $\mathbb{R}^2\setminus\{(0, 0)\}$ and thus, the CR equations prevent $f$ from being complex differentiable anywhere else.
			%
			\item Real differentiable everywhere.\\
			Complex differentiable precisely at $0.$\\
			Holomorphic nowhere.\\~\\
			Same steps as above.
			%
			\item Real differentiable everywhere. Complex differentiable nowhere. Use CR equations again.
			%
			\item 
			$f$ is real differentiable precisely on $\mathbb{R}^2\setminus\{(0, 0)\}.$\\
			However, it is not complex differentiable anywhere.\\~\\
			Breaking as earlier, we get
			\begin{equation*} 
				u(x, y) = \dfrac{x^2 - y^2}{x^2 + y^2}, \quad v(x, y) = \dfrac{2xy}{x^2 + y^2},
			\end{equation*}
			for $(x, y) \in \mathbb{R}^2\setminus\{(0, 0)\}$ and
			\begin{equation*} 
				u(0, 0) = 0 = v(0, 0).
			\end{equation*}
			Note that $u$ and $v$ aren't even continuous at $(0, 0).$ Thus, neither if $f.$ Hence, $f$ is neither real nor complex differentiable at $(0, 0).$ \\
			However, at all other points, all partial derivatives exist and are continuous. Thus, $f$ is real differentiable at all those points. However, computing $u_x, u_y, v_x, v_y$ explicitly shows that the CR equations are not satisfied anywhere. Thus, $f$ is not complex differentiable anywhere. \qedhere
		\end{enumerate}
	\end{soln}
	%
	%
	%
	\item  Show that the CR equations take the form
	\begin{equation*} 
		u_r = \dfrac{1}{r}v_\theta, \quad v_r = -\dfrac{1}{r}u_\theta
	\end{equation*}
	in polar coordinates.
	%\\
	\begin{soln}
		We shall follow the same idea as in the slides. We first write
		\begin{equation*} 
			f(r, \theta) = f(re^{\iota\theta}) = u(r, \theta) + \iota v(r, \theta).
		\end{equation*}
		Suppose that $f$ is differentiable at $z_0 = r_0e^{\iota\theta_0} \neq 0.$ (Note that it wouldn't make sense to talk at $0$ since there's a $r^{-1}$ factor in the question anyway.)\\
		Thus, we know that the limit
		\begin{equation*} 
			\lim_{z\to z_0}\dfrac{f(z) - f(z_0)}{z - z_0}
		\end{equation*}
		exists. We shall calculate it in two ways:
		\begin{enumerate}
			\item Fix $\theta = \theta_0$ and let $r \to r_0.$ Then, we get
			\begin{align*} 
				f'(z_0) &= \lim_{r\to r_0}\left\{\dfrac{u(r, \theta_0) - u(r_0, \theta_0)}{e^{\iota\theta_0}(r - r_0)} + \iota\dfrac{v(r, \theta_0) - v(r_0, \theta_0)}{e^{\iota\theta_0}(r - r_0)}\right\}\\~\\
				&= e^{-\iota\theta_0}\lim_{r\to r_0}\left\{\dfrac{u(r, \theta_0) - u(r_0, \theta_0)}{r - r_0} + \iota\dfrac{v(r, \theta_0) - v(r_0, \theta_0)}{r - r_0}\right\}\\~\\
				&= e^{-\iota\theta_0}\left(u_r(r_0, \theta_0) + \iota v_r(r_0, \theta_0)\right). & (*)
			\end{align*}

		\item Fix $r = r_0$ and let $\theta \to \theta_0.$ Then, we get
		\begin{align*} 
			f'(z_0) &= \lim_{\theta\to \theta_0}\left\{\dfrac{u(r_0, \theta) - u(r_0, \theta_0)}{r_0(e^{\iota\theta} - e^{\iota\theta_0})} + \iota\dfrac{v(r_0, \theta) - v(r_0, \theta_0)}{r_0(e^{\iota\theta} - e^{\iota\theta_0})}\right\}\\~\\
			&= \dfrac{1}{r_0}\lim_{\theta\to \theta_0}\left\{\dfrac{u(r_0, \theta) - u(r_0, \theta_0)}{e^{\iota\theta} - e^{\iota\theta_0}} + \iota\dfrac{v(r_0, \theta) - v(r_0, \theta_0)}{e^{\iota\theta} - e^{\iota\theta_0}}\right\} & (**)
		\end{align*}
		We concentrate on the first term of the limit. Note that
		\begin{align*} 
			&\lim_{\theta\to \theta_0}\dfrac{u(r_0, \theta) - u(r_0, \theta_0)}{e^{\iota\theta} - e^{\iota\theta_0}}\\~\\
			=& \lim_{\theta\to \theta_0}\dfrac{u(r_0, \theta) - u(r_0, \theta_0)}{\theta - \theta_0}\dfrac{\theta - \theta_0}{e^{\iota\theta} - e^{\iota\theta_0}}.
		\end{align*}
		In the product, the first term is clearly $u_\theta(r_0, \theta_0),$ after taking the limit. The second term can be calculated to be
		\begin{equation*} 
			\dfrac{1}{\iota e^{\iota\theta_0}}.
		\end{equation*}
		(How? Write $e^{\iota\theta}$ in terms of $\cos$ and $\sin$ and differentiate those and put it back.)\\
		Of course, a similar argument goes through for the $v$ term as well.\\
		Thus, we get that $(**)$ transforms to
		\begin{equation*} 
			f'(z_0) = \dfrac{e^{-\iota\theta_0}}{r_0}\left(\iota u_\theta(r_0, \theta_0) + v_\theta(r_0, \theta_0)\right).
		\end{equation*}
		\end{enumerate}
		Equating the above with $(*),$ cancelling $e^{-\iota\theta_0},$ and comparing the real and imaginary parts, we get
		\begin{equation*} 
			u_r(r_0, \theta_0) = \dfrac{1}{r_0}v_\theta(r_0, \theta_0), \quad v_r(r_0, \theta_0) = -\dfrac{1}{r_0}u_\theta(r_0, \theta_0),
		\end{equation*}
		as desired.
	\end{soln}
\end{enumerate}
\end{document}